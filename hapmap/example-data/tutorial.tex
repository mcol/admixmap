\documentclass[preprint, a4paper]{aastex}

\usepackage{amsmath} \usepackage{amsfonts} \usepackage{amssymb}
 \usepackage{natbib} 

\title{Using HAPMIXMAP to test all loci in a
   candidate gene in which tag SNPs have been typed in cases and
   controls} \author{Paul M.  McKeigue}

\begin{document}

\section{Data files}
The dataset consists of a case-control study in which 1000 individuals
of  European ancestry  were typed  at 9  SNPs in  a candidate  gene on
chromosome 7.  The SNPs were chosen as tag
SNPs,  using the  program TAGGER,  based on  an early  version  of the
HapMap.  The raw data files are genotypes.txt, and outcome.txt 

To prepare  the files for  analysis by HAPMIXMAP, the  following steps
are required
\begin{itemize}
\item{}Download the HapMap haploid data file for the chromosome containing
  the  SNPs  and  the  continental  group  that  was  sampled  in  the
  case-control study
\item{} Extract  all hapmap data  for the region containing  the typed
  SNPs, including 100 kb on either side of the sequence of typed loci,
  and write out  this file in HAPMIXMAP format.   For this example, we
  obtain a file of 298 loci spanning 160 kb. 
\item{} Write a table of these loci in HAPMIXMAP format.  
\item{} Drop any loci from the raw genotypes file that are not matched
  in the HapMap.  Recode the alleles in the raw genotypes file so that the
numeric coding corresponds  to that in the HapMap  database, sort by
  map position, and write file genotypes.txt in HAPMIXMAP format
\end{itemize}

\section{Running the analysis}
To reduce the time taken for the program to converge, we undertake an
initial run using only the phased HAPMAP haplotypes and save a single
draw from the posterior distribution of model parameters.  The saved
values are used as initial values for the subsequent run in which the
posterior distribution is generated given both HAPMAP haplotypes and
diploid case-control genotypes at tag SNPs.

For the first run, type

<path to hapmixmap executable>/hapmixmap training-initial.conf

For subsequent runs, type

<path to hapmixmap executable>/hapmixmap training-resume.conf

Repeat  until  the  posterior   mean  of  the  ``energy''  (minus  the
log-likelihood of the model parameters given the data) is no longer falling
with each new run.  

For the case-control analysis, type 

<path to hapmixmap executable>/hapmixmap testing.conf

Finally, type 

perl post-process.pl

to run an R script that processes the files written by HAPMIXMAP


\section{Interpretation of results}
The results are written to the folder /results below the current
working directory.  To determine whether the program has been run with
a long enough burn-in, examine a plot of the realized values of
``energy'' (minus the log-likelihood of the model parameters given the
data) that are in the file loglikelihoodfile.txt; if there is a
downward trend, the burn-in was too short.  Another check is to look
at the Geweke diagnostics for the global model parameters in the file
ConvergenceDiagnostics.txt: for each parameter, the test compares the
mean over the first 10\% of iterations (after the burn-in) with the
mean over the last 50\% of iterations.

To determine whether the program has been run long enough for the
score tests to be computed accurately, open the file
TestAllelicAssociations.ps with a postscript viewer and examine the
plot of log $p$-values based on all iterations so far.  Towards the
end of the run, the plots should be nearly horizontal lines.

The file PosteriorQuantiles.txt contains the posterior means, medians
and 95\% credible intervals for the global model parameters.  The
average number of arrivals per megabase (a) is inferred as 56.2.  With
eight hidden states, 7/8 of these arrivals will change the block
state.  We can therefore calculate the estimated mean haplotype block
length in this region as 20 kb ($1000 \times 8 / \left[ 7 \times a \right]$) in this
region, rather shorter than the genomewide average of 27 kb in
European-ancestry  populations.  This may  be one  reason why  the tag
SNPs perform poorly in predicting genotypes at untyped loci.  

The results of the tests for associations are contained in the file
AllelicAssociationTestsFinal.txt.  This has one row per locus.  The
column labels are explained below
\begin{itemize}
\item{Score (U)}This is the gradient of the log-likelihood as a
  function of the regression parameter $\beta$ for the effect of
  allele 2 (coded as 0, 1, 2 copies) at the null value $\beta=0$.  For
  a case-control study, the regression model is a logistic regression,
  with disease status as outcome variable and any specified variables
  as covariates.
\item{Complete information } This can be interpreted as a measure of
  how much information about $\beta$ you would have if the locus had
  been typed directly - where the complete information is small, this
  is because the locus is not very polymorphic
\item{Observed information (V)} This is minus the second derivative of
  the log-likelihood function at $\beta = 0$.  In large samples, the
  log-likelihood function is approximately quadratic and the maximum
  likelihood estimate of $\beta$ is therefore approximately $U/V$.
  This approximation only holds where the observed information is
  reasonably large.
\item{Percent information } This can be interpreted as a measure of
  the efficiency of the tag SNP panel in extracting information about
  the effect of the locus under study
\item{Missing 1} This can be interpreted as the percent of information
  that is missing because the sample size of the HapMap panel is small
  SNP panel
\item{Percent missing 2} This can be interpreted as the percent of
  information that is missing because the tag SNP panel is inadequate
\item{Standard normal deviate} This is $U / \sqrt{V}$.  This value
  will not be computed where the the observed information is small.
  Where there is not enough information, the asymptotic properties of
  the score test do not hold.
\item{$p$-value } Two-sided $p$-value for the standard normal deviate
\end{itemize}

A plot of the percent information against map position, or inspection
of the table, shows that the observed information is about 90\% at the
9 typed loci, but much lower at the other untyped loci.  In other
words, our tag SNP selection did not do a very good job of predicting
the genotypes at untyped loci.  Towards the ends of the region
included in the model, the observed information is close to zero
(equivalent to a flat likelihood function).  This is to be expected,
as we have no tag SNPs typed in this region.

\end{document}